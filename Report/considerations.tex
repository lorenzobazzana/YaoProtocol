\section{SEL considerations}\label{sec:sel}

\subsection{Advantages of MPC to Society}

The large scale applications of MPC could greatly benefit society as a whole: the world is moving in a direction where IoT and the Cloud are more and more important for everyday life applications. Having the possibility of preserving privacy even when computing shared data is crucial in such a world. MPC could also improve fields where massive data is required, like Machine Learning: being able to train Machine Learning models using data from private users whilst maintaining privacy could be very beneficial for the accuracy of the model, because it would be trained on a wide and varied dataset.

Another consequence of large MPC applications is collaboration: data is a valuable resource, and for this reason companies and governments are reluctant to share it; MPC instead allows collaboration without revealing the data, so it could open new ways of collaboration between parties at different levels.

\subsection{Ethical implications}

MPC's usage in a global setting must guarantee that these techniques are secure even in an adversarial environment. Malicious parties could leverage on the privacy given by MPC for their advantage, be it trying to learn information on other users' data, or using the shared computational power in some other form. For example, multinational companies could use MPC for statistical market analysis using their shared but private data, and using the results to influence the customers or other forms of market manipulation.

From an ethical point of view then these techniques must be used carefully, especially in a context where the legitimacy of the computation results is very important (for example, e-voting). Another aspect involved with ethical and moral questions is data management: MPC brings new ways of treating data, changing the notion of responsibility and accountability of and for that data. Consequently, we must define these concepts in a way that also considers the use of data in MPC, and at the same time ensure that MPC respects those definitions.

\subsection{Legal aspects of MPC}

The definitions of responsibility and accountability mentioned in the previous paragraph are also deeply connected to the legal aspect of privacy regulations, as in order to guarantee a fair usage of MPC it must comply with all the data and privacy regulations (like the european GDPR). The usage and characteristics of MPC implementations should also be regulated, so that in a commercial setting the users/customers are legally protected against eventual misuse of these technologies (like in the case of some party trying to recover information about the users' data).

Other important legal aspects of data usage that come into play when using MPC are data ownership and intellectual property: even if the data used in MPC is private, it still influences the result of the computation. Because of this it is crucial to clearly define frameworks to evaluate who owns the resulting data, and for what means it can be used.